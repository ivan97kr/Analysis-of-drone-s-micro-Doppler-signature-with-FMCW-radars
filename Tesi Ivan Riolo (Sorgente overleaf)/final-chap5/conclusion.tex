\chapter{Conclusions and future works}
%Scrivire un report finale:
%1. punto di partenza e obiettivo della tesi:
% (indagare analisi micro Doppler per identificare un drone, approccio classico, nuovo approccio sul Range-Profile)
%2. Cosa ho costruito con la mia tesi? (Modello di segnale 'beaten' ricevuto dal drone, modello post-processing radar FMCW 2DFFT con STFT nella seconda trasformata, algoritmo di supporto alla scelta della forma d'onda migliore date le condizioni di risoluzione desiderata.
%3.Compatibilità del modello di segnale e di processing con spettrogrmami costruiti con altri modelli di segnali, valido teoricamente. Prestazioni raggiunte, livello di precisione delle misure con quadro sulle caratterisiche del radar necessario nel caso range profile e spettrogramm. Conclusione principale, si può fare target idetification dei droni analizzando la mD sul range profile? Si! Ma... approfondimento sui parametri del radar in questo caso e sulla visibilità dei vari droni.
%in quali categorie può essere fatta identification?
%Improvements? e sviluppi futuri
The starting point of the thesis was to investigate the current solutions for drone detection and identification. It was seen that radar solutions are the best performing in this case but also the most expensive from an energy point of view considering the very small size of drones. The main objective of the thesis was therefore to investigate possible radar solutions and find a potentially 'low cost' one that is portable and applicable to 'civil' contexts. Micro Doppler analysis has proven to be the most effective way to identify a drone by means of a radar, as the spectral signature of a drone is unique and distinguishable from any other flying object or bird. The main distinction that needs to be made is with regard to birds, which are often present and are the same size as a drone. Therefore, in order to avoid very frequent false detections, it is necessary to be able to distinguish the two subjects.
With the unique waveform designed for the two different analyses (Range-Time and Frequency-Time), it is certainly possible to make this coarse distinction. Not only that, as we have seen, it is possible to derive features about the drones such as blade length and rotation speed, which allows to identify the model. Although in the objective scenarios of this thesis, it is not necessary to identify the type of drone and its model, it is sufficient to distinguish it from other objects. It could be useful to derive the specific drone model in other military-level applications where it is necessary to understand which drone it is and then apply the appropriate countermeasures.\\
The most widely studied micro Doppler analysis in the literature is carried on the drone's spectrogram, while the study conducted in this thesis shows that it is possible to observe the same drone's features in the drone's range profile at a lower cost. The starting point of the Range-Time analysis was the cell migration effect that typically occurs in wideband radar when the chirp duration is long enough. Usually it is wanted to avoid this effect to conduct the Frequency-Time analysis, while if it is emphasized the main result of this thesis are obtained. Then was shown that using a frequency modulated wideband radar, i.e FMCW radar, provides a device capable of performing both analyses and the only difference is the waveform that is specific in the two cases. Unlike a simple continuous wave radar, which only allows Frequency-Time analysis without the possibility of measuring distances.\\
Having identified the FMCW radar as a possible solution to low-cost micro Doppler analysis, a received signal model from rotor's blades compatible with the models already known and studied in literature was identified. A very important result is that the signal model used is compatible with these. Subsequently, the classic 2DFFT procedure carried out in the FMCW radar was adapted to the case where first a range-time map and then a spectrogram is needed.\\
Two different algorithms to support the selection of the best waveform were designed depending on the type of analysis, in the case where the migration effect is to be emphasised and exploited as a benefit to conduct the analysis on the Range-Time plane or in the opposite case where migration effect must be avoided and treated as a disturbance in order to conduct the analysis on the Freqeuncy-Time plane.\\
With the designed FMCW radar, in addition to the possibility of performing both analyses, there are very low costs when choosing to use the designed range-time waveform. In fact, the required sampling frequency in this case is only $1.55 MHz$ with an average transmitting power of about $6 Watts$. Under these conditions, it is possible to measure all the features of a drone helicopter with millimetre accuracy. So, not only a drone helicopter can be distinguished from any other object or bird in a coarse way, but it is possible to identify the specific model. In the case of qudcopter drone, it is possible to measure all features with the same precision if the initial phase of the different rotors allow this. On the other hand, it is not possible to measure the number of rotors present as the temporal resolution in the Range-Time plane is lower than in the Frequency-Time plane. Another important benefit of Range-Time analysis, mentioned in the previous chapter, is the processing time required. Since it is not required to compute the spectrogram but only to perform the FFT of each received echo, which is a fairly quick task and can be performed in parallel once all echoes have been received.\\
A unique waveform that meets the resolution requirements for both quadcopter and helicopter drones is one of the main achievements which makes the device less complex and easier to implement.\\ Considering the need for low cost and portability in reference scenarios, such as a crowded place or a private house, the Range-Time analysis fully meets these requirements and certainly allows to distinguish a drone from other objects or birds. It also certainly allows to distinguish a helicopter drone from a quadcopter drone. More specifically, allows the possibility to retrieve the helicopter drone model while only in lucky cases allows to retrieve the quadcopter drone model. The only performance difference with classical spectrogram analysis is the temporal resolution, which in Range-Time plane does not allow for the high levels needed to visualize every single blade flash of a drone with multiple rotors. But since it is not necessary to identify the specific drone model in these scenarios, it is possible to accept this trade-off in favor of lower power requirements, less expensive components such as the ADC converter, and a shorter processing time.\\
The HERM (Helicopter Rotor Modulation) spectral analysis, described in \cite{HERM}, is another possible 'low-cost' way to work in speed ambiguity, as is the case in range-time analysis, could be investigated in the future and be compared to the solution proposed in this thesis. 
Another important future development is in cases where transmitted power and cost are not a stringent limitation and frequency-time analysis can be improved and exploited to precisely identify observed drone model, specifically for the quadcopter drones that are more difficult to identify due to their lower blade flash period due to the presence of more rotors.
