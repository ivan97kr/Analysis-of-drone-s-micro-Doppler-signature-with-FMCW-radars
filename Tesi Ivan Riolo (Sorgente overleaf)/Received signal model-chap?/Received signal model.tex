\chapter{Drone's received signal model}
In this chapter the treatment made to extrapolate a correct model for the signal received from a drone is derived in stages. Starting from the signal received by a rotating object seen in the previous chapter, we move on to represent the signal received by rotating blades. Considering in succession the presence of $N_r$ rotors, each of which is in a different position from the center of mass and each of which has $N_b$ blades. In such a way as to model in the best way the signal received by a drone over time. In the previous chapter the effect of micro Doppler in the case of narrowband or a wideband radar was analysed in depth. It has been seen that in the case of wideband it is appropriate to take into consideration some terms and exponents referring to the fast time which are not negligible and cause the cell migration effect. By putting oneself in such conditions that the migration effect is negligible, even the case of a wideband radar can be treated with the simpler model of the narrowband radar, derived from the Stop-go model. 
So, the use of a wideband radar is considered, as in the case of our interest for an FMCW radar. In the end different models of backscattered signal from a drone are analyzed and compared, deciding which is the most suitable for the radar case under consideration.


\section{Received signal model from rotor blades}
Considering that the problem of range cell migration discussed in the previous chapter has been solved, the signal received by a rotating blade is now described.
Using an FMCW radar with an appropriate combination of parameters so that the narrowband model can be used to describe a wideband radar as well. Thus based on the Stop-go model whose reference equation is the \ref{stopgomodelformula}.